%!TEX root = ../main.tex
\documentclass[../main]{subfiles}
\begin{document}

  \chapter{Model Equations}\label{chapter:modelequations}

  In this chapter we build on \autoref{chapter:sizetheory} to construct a fully derived equation to model in \autoref{chapter:method}. We begin by considering numerical interpretations of ``feeding kernels'' and prey selection, then construct the various predation coefficients for \autoref{theory:eq:jumpdiffusion}.

  \section{Feeding Kernels}
  Central to the models that we are studying is the idea of a feeding kernel which is the rate at which a predator eats prey of a different size. While there are many interpretations of feeding kernels we use an interpretation centralised around the idea that `big fish eat smaller fish'. We assume that the likelihood a predators of weight $w$ feeds upon prey with weight $w'$ is based upon both theses weights.

  Based upon this key idea \cite{benoit2004} showed that the feeding kernel for an individual takes the form

  \begin{equation}\label{model:eq:kernel}
    k(w, w') = s w^{\alpha} S\left( \frac{w}{w'} \right)
  \end{equation}

  for some parameters $s, \alpha$ to be determined and for the size selection function $s$. This states the rate at which a predator of weight $w$ feeds is a product of the volumetric search rate $s w^{\alpha}$ and the feeding preference function $S$. In this thesis we make the assumption that $s = 1$ for convenience.

  \subsection{Prey Selection}
  As discussed the prey selection will peak around some fixed ratio $w / w'$ for predator / prey, which we call $\beta$. Assuming there is some spread for which individuals will vary from this ratio, taken as $\sigma$, then we can say that the prey selection function $S$ will take the form

  \begin{equation}
    S(z) = \kappa \exp{\left( - \frac{(z - \beta)^2}{2 \sigma^2} \right)}
  \end{equation}

  for some $\kappa$ that we discuss later.

  \subsection{Coefficient Functions}
  When we consider the transport equation form of the McKendrick-von Foerster we can extract coefficients. Taking \autoref{theory:eq:jumpdiffusion} in the form of \autoref{theory:eq:jumptransport} it reads

  \begin{align}\label{model:eq:jumpdiffusion}
    \frac{\partial u}{\partial t} =
    & - \left( \left( \int (K w') k(w, w') u(w') \: \mathrm{d}w' \right) \cdot u \right)_w \nonumber \\
    & -  \left( \int k(w', w) u(w') \: \mathrm{d}w' \right) \cdot u \nonumber \\
    & + \frac{1}{2} \left( \left( \int (K w')^2 k(w, w') u(w') \: \mathrm{d}w' \right) \cdot u\right)_{xx}.
  \end{align}

  Consider the growth term in \autoref{model:eq:jumpdiffusion},

  \begin{equation}\label{model:eq:growth}
    g(w) = \int (K w') k(w, w') u(w'),
  \end{equation}

  we can think of this as the expected value of the mass eaten by individuals of weight $w$. Thus the prey selection function can be thought of as a probability distribution and so we take $$\kappa = \frac{1}{\sigma\sqrt{2 \pi}}.$$

  If we take \autoref{model:eq:growth} and let $\upsilon = g$, and define

  \begin{eqnarray}
    R = \mu(w)  &=& \int k(w', w) u(w') \: \mathrm{d}w', \\
    D(w)    &=& \int (K w')^2 k(w, w') w(w') \: \mathrm{d}w'
  \end{eqnarray}

  we have \autoref{theory:eq:jumpdiffusion} in the form of \autoref{theory:eq:transport1d}.

  \section{Steady State Solutions}
  In marine ecosystems it is been found that the abundance of organisms within weight classes is roughly constant (\cite{sheldon1972}) if these weight classes are distributed logarithmically. Further, averaged over time, \cite{datta2011} observes that this abundance changes little suggesting the system is near steady state.

  \cite{benoit2004} found that the McKendrick-von Foerster Equation has a power law steady state of the form $\phi(w) \propto w^{\gamma}$, for $\gamma \in \mathbb{R}^-$. We note that in the real world such a power law steady state must break down since it predicts an infinite number of individuals with weight $w = 0$.

  Combining this research to show that \autoref{model:eq:jumpdiffusion} can be transformed using the logarithmic change of variables and appropriate change of function to have a constant solution as \cite{benoit2004} discusses.

  \subsection{Logarithmic Scale}
  \cite{sheldon1967}, who first introduced the idea of a size spectrum to organise counts of particles in the ocean, stipulates the fundamental conjecture that the total mass within logarithmically spaced size groups was constant over the size range from bacteria to whales.

  Interpreting this conjecture biologically \cite{silvert1980} poses that the total mass of prey is the same as the mass of their predators, but can be written in terms of the number spectrum, where the number of individuals with size $m$ per volume is described by a density function $N(w) \propto w^{-\gamma}$. Which has later been confirmed by \cite{benoit2004}.

  Since these conjectures assume a logarithmic weight scale, we perform a change of variables from \autoref{model:eq:jumpdiffusion} using the dimensionless variable $x = \log{w}$ assuming that $\phi(x) \mathrm{d}x = u(w) \mathrm{d}w$. Following \autoref{theory:eq:jumptransport} we have

  \begin{equation}
    u_t = - (g \cdot u)_w - \mu \cdot u + \frac{1}{2} (D \cdot u)_{ww}
  \end{equation}

  then after changing variables using a logarithmic weight scale we gain a new equation for $\phi$ reading

  \begin{equation} \label{model:eq:logscale}
    \phi_t = - (\hat{g} \cdot \phi)_x - \hat{\mu} \cdot \phi + \frac{\e{-x}}{2} \left( \left( \hat{D} \cdot \phi \right)_{xx} - \left( \hat{D} \cdot \phi \right)_x \right).
  \end{equation}

  However, if we are to maintain a constant steady state as stipulated by \cite{sheldon1967}, we must perform a simple change of variables. The steady state $u(w) \propto w^{\gamma}$ reduces to $\phi(x) \propto \e{x(\gamma +1)}$, which is clearly not constant. If we reduce to a constant by taking $\e{\gamma x} \rho(x) = \phi(x)$ then we have our constant but $\rho$ is governed by the more complicated equation

  \begin{equation} \label{model:eq:logconstanteq}
    \e{\gamma x} \rho_t = - (\bar{g} \cdot \rho)_x - \bar{\mu} \e{\gamma x} \rho + \frac{\e{-x}}{2} \left( \left( \bar{D} \cdot \rho \right)_{xx} - \left( \bar{D} \cdot \rho \right)_x \right).
  \end{equation}

  for transformed coefficients $\bar{g}, \bar{\mu}, \bar{D}$.

  \subsection{Constant Steady State}
  Following the change of variables $\e{\gamma x} \rho(x) \mathrm{d}x = u(w) \mathrm{d}w$ we find that the coefficients of \autoref{model:eq:logconstanteq} read
  \begin{eqnarray}
    \bar{g}(x)  &=& \int K \e{x' + (\alpha + \gamma - 1) x} S(\e{x - x'}) \e{\gamma x'} \rho(x') \: \mathrm{d}x' \\
    \bar{d}(x)  &=& \int K^2 \e{2x' + (\alpha + \gamma - 1) x} S(\e{x - x'}) \e{\gamma x'} \rho(x') \: \mathrm{d}x' \\
    \bar{\mu}(x) &=& \int \e{\alpha x'} S(\e{x' - x}) \e{\gamma x'} \rho(x') \: \mathrm{d}x'.
  \end{eqnarray}

  Taking the same assumption as \cite{datta2011} that $\alpha + \gamma - 1 = 0$ and substituting $\rho(x) = C$ as the steady state, we can find an equation for $\alpha$ which will guarantee a constant steady state. Cancelling some factors and taking the convention that $r = x - w'$ the steady state condition for the McKendrick-von Foerster Equation with diffusion reads

  \begin{equation} \label{model:eq:steadystatecondition}
    0 = \int S(\e{r}) \left(-1 + \alpha K \e{(\alpha + 1)r} + + \alpha (\alpha + 1) \frac{K}{2} \e{(\alpha + 2)r} \right) \: \mathrm{d}r
  \end{equation}

  while if we do not include the diffusion term, we can ignore the $K^2$ terms in \autoref{model:eq:steadystatecondition}.

  \section{Summary}
  In this chapter we have outlined the realised equations that will be used in \autoref{chapter:method}. Next we will discuss some of the methods that we can use to approach these equations analytically and numerically.


\end{document}
