%!TEX root = ../main.tex
\documentclass[../main]{subfiles}

\begin{document}
  \begin{titlepage}
    \BgThispage
    \newgeometry{left=1.5cm,right=6cm,bottom=2cm}
    \vspace*{0.3\textheight}
    \noindent
    \printtitle
    \vspace*{2cm}\par
    \begin{minipage}{1.1\linewidth}
      \begin{minipage}[c]{0.25\linewidth}
        \begin{flushright}
          \printauthor
        \end{flushright}
      \end{minipage} \hspace{15pt}
      %
      \begin{minipage}{0.02\linewidth}
        \rule{1pt}{225pt}
      \end{minipage} \hspace{25pt}
      %
      \begin{minipage}[c]{0.73\linewidth}

      \end{minipage}
    \end{minipage}
    \noindent
  \end{titlepage}
  \restoregeometry

  \onehalfspacing
  \setstretch{1.5}
  \pagenumbering{roman}

  \renewcommand{\abstractname}{\Huge Abstract}
  \begin{abstract}
    \addcontentsline{toc}{chapter}{Abstract}
    \thispagestyle{plain}
    \raggedright
    While the mechanics of marine ecosystems have been researched for years, recent efforts by \cite{datta2010} and \cite{hartvig2011} derive fully non-linear equations to describe their mechanics. Unfortunately these equations are difficult to solve and require numerical methods to solve and methods with a high order of approximation are yet to be developed thoroughly.

    This project investigates the derivation of the equations used to model marine ecosystems, in particular the Jump-Growth Equation  and various forms of the McKendrick-von Foerster Equation. It shows that particular forms of the McKendrick-von Foerster Equation are approximations to the Jump-Growth Equation and construct the coefficients necessary to approximate the Jump-Growth Equation. Using the second order approximation we show that, for particular parameters, this approximation satisfies a power law steady state.

    This project considers numerical methods for solving these equations including the history of finite differences and the construction of accurate approximations. It contains a full discussion of the errors generated by using finite difference approximations and how to resolve the issues that arise from using them. This is helpful when it considers applying these methods to the McKendrick-von Foerster Equation. It shows that while the simple forms of the McKendrick-von Foerster Equation are easy to solve numerically, more work is needed before accurate methods can be used on the non-linear forms of the McKendrick-von Foerster Equation.
  \end{abstract}

  % \renewcommand{\abstractname}{\Huge Acknowledgements}
  % \begin{abstract}
  %   \setcounter{page}{2}
  %   \addcontentsline{toc}{chapter}{Acknowledgements}
  %   \thispagestyle{plain}
  %   Thanks Mum!
  % \end{abstract}

  \setlength{\parskip}{0em}
  \setcounter{page}{3}
  \tableofcontents
  \listoffigures

  \chapter{Introduction}\label{chapter:introduction}
  % Set Options for Main Document
  \setstretch{1.5}
  \pagenumbering{arabic}

  The study of partial differential equations has always been of interest due to their close relation to modelling real world ecosystems. In particular a class of partial differential equations called transport equations relate real world convection-diffusion systems to analytic equations and are studied extremely widely. In this project we study the details around one particular transport equation: the McKendrick-von Foerster Equation.

  \begin{equation} \label{intro:eq:mvf}
    \frac{\partial u(t, w)}{\partial t} = - \frac{\partial}{\partial w} \left(g(w) \cdot u(t, w) \right) - \mu(w) \cdot u(t, w)
  \end{equation}

  \section*{Project Outline}
  This project investigates the current methods used in research around \autoref{intro:eq:mvf} and similar equations, focussing on trying to derive an higher order finite difference equation for the Jump-Growth Equation

  \begin{align}\label{intro:eq:jumpgrowth}
    \frac{\partial \phi(w)}{\partial t}
    = \int ( &- k(w', w) \phi(w)\phi(w') \nonumber \\
    & - k(w, w')\phi(w')\phi(w) \nonumber \\
    & + k(w - Kw', w')\phi(w - Kw')\phi(w')) \: \mathrm{d}w'.
  \end{align}

  In \autoref{chapter:sizetheory} we discuss the historical background and derivation of the McKendrick-von Foerster Equation. We study similarities between the equation transport equations which will serve as the basis for our approach towards constructing finite difference approximations for the equations in \autoref{chapter:method}. Lastly we give a short overview of the Jump-Growth Equation derived by \cite{datta2010}. We show that the McKendrick-von Foerster Equation with diffusion is an approximation of the Jump-Growth Equation and focus on the simpler equation given by this approximation.

  \autoref{chapter:modelequations} constructs the model equations that will be used in \autoref{chapter:method} by considering the derivation of the coefficients in the Jump-Growth Equation. In the second half of the chapter we consider the steady states of these equations which will be useful when analysing the solutions to the numerical approximations as we can consider how close to the steady states the solutions are.

  Before performing the final calculations and discussion of approximating the Jump-Growth Equation and the McKendrick-von Foerster Equation in \autoref{chapter:method}, we introduce the mathematical framework of finite difference equations in \autoref{chapter:fdes}. We discuss the history of difference equations and their application to the simple model problem of the Advection Equation

  \begin{equation}
    \frac{\partial u}{\partial t} + v \frac{\partial u}{\partial x} = d u.
  \end{equation}

  Finally we discuss how to handle the errors generated by approximating the equations derived in \autoref{chapter:modelequations} by finite difference equations and how we aim to minimise this error so that the approximations converge correctly.

\end{document}
